\documentclass[fontsize=12pt,paper=a4,DIN]{scrlttr2}
\usepackage[T1]{fontenc}
\usepackage[ngerman]{babel}
\usepackage[utf8]{inputenc}

% Setzen der Absenderadresse
\setkomavar{fromaddress}{%
Frank Lanitz
Mühlenstraße 9
07745 Jena}
%\setkomavar{fromname}{Hackspace Jena e.V.}
\setkomavar{fromname}{Frank Lanitz}
\setkomavar{fromemail}{frank@frank.uvena.de}
\setkomavar{fromphone}{+49 (0)152 047 352 63}
\setkomavar{place}{Jena}
\setkomavar{signature}{Frank Lanitz}

%KOMAscript schön machen
\KOMAoptions{headsepline,%
pagenumber=topright,%
fromalign=center,%
fromrule=on,%
foldmarks=off,%
backaddress=off,%
firstfoot=off,
locfield=wide,%
enlargefirstpage=on,%
parskip=half%
}

\addtokomafont{fromname}{\Large}
\makeatletter
\@setplength{locwidth}{.5\textwidth}
\makeatother

\setkomavar{location}{\raggedleft
  \usekomavar*{fromemail}\usekomavar{fromemail}\\
  \usekomavar*{fromphone}\usekomavar{fromphone}\par
}

\date{7. Dezember 2011}

\begin{document}

\begin{letter}{Vereinsregister \\ 
	z.Hd. Frau Stammberge \\ 
	Rathenaustraße 13 \\ 
	07745 Jena}

\setkomavar{subject}{Anfrage bzgl. Vorabprüfung der Satzung für 
Vereinsgründung}

\opening{Sehr geehrte Frau Stammberger, sehr geehrte Damen und Herren,}

Anbei übersende ich Ihnen wie telefonisch besprochen den aktuellen 
Entwurf der Satzung des Hackspace Jena sowie die Geschäftsordnung 
zur Vorabprüfung für eine Eintragung in das Vereinsregi\-ster.

Um bereits im Vorhinein die Frage nach dem Vereinssitz zu klären, 
planen wir auf Grund der aktuell noch fehlenden, dauerhaften 
Räumlichkeit, die Adresse des Vorstandsvorsitzenden als Vereinssitz 
zu nutzen. 

Weiter planen wir die konstituierende Versammlung im Januar 2012 
durchzuführen und im Anschluss, Anfang Februar, die Eintragung in 
das Vereinsregister zu beantragen. Aus diesem Grund würden wir uns, 
um im Falle von Problemen noch ein wenig Zeit zur Korrektur zu haben,
über eine kurze Rückmeldung bis Ende des Jahres freuen. 


\closing{Mit freundlichen Grüßen}

\end{letter}
\end{document}
