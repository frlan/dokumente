\documentclass[10pt, a4paper]{scrartcl}
\usepackage[ngerman]{babel}
\usepackage[utf8]{inputenc}
\begin{document}

\section*{Geschäfts- und Tätigkeitsbericht 2012}
Der Verein Hackspace-Jena e.\,V. ist in das Vereinsregister eingetragen wurden und
es sind Vereinskonto und Postfach angemeldet wurden.
Die Satzung wurde im Hinblick auf die Erlangung der Gemeinnützigkeit überarbeitet.
Der Verein hat sich durch Vorträge z.\,B. an der Fachhochschule vorgestellt.

Es sind Büroräume in der Krautgasse angemietet, eingerichtet und bezogen wurden,
diese erhielten später den Namen Krautspace.
Sie beherbergen ein Elektroniklabor und Platz zur Durchführung von Workshops und Vorträgen,
dies wird durch die Einrichtung wie Whiteboard,
Fachbuchbibliothek zum Thema Informations- und Elektrotechnick und Internetanschluss unterstützt.
Daneben wurde eine wechselnde Fotoausstellung etabliert.
Durch eine selbst entworfene und umgesetzte Türschließanlage stehen die Räume Mitgliedern zur Verfügung. 

\subsection*{Regelmäßige Vereinsaktivitäten}
Folgende Aktivitäten fanden Wöchentlich bzw.\ Monatlich statt:
\begin{itemize}
	\item Offene Dienstagsrunde
		\begin{itemize}
			\item Austausch über Computersicherheit, Datenschutz und Informatik
			\item oft spontane Kurzvorträge
		\end{itemize}
	\item Chaostreff
		\begin{itemize}
			\item Diskussion über Ereignisse aus dem Umfeld des Chaos Computer Clubs
		\end{itemize}
	\item Plenum
		\begin{itemize}
			\item Diskussion aktueller Vereinsthemen
			\item Erarbeitung von Entscheidungshilfen für den Vorstand
		\end{itemize}
	\item Stammtisches der Linux-User-Group Jena
		\begin{itemize}
			\item Erfahrungsaustausch, gegenseitigen Hilfe bei Problemen zum Thema freie Software und insbesondere GNU/Linux
		\end{itemize}
	\item Gemeinsames Kochen
		\begin{itemize}
			\item Unkompliziertes Zusammentreffen zu einem weniger technischen Thema
			\item Vermittlung von kulinarischen Fähigkeiten und Kreativität in der Küche
		\end{itemize}
\end{itemize}

\subsection*{Vorträge \& Workshops}
Eine Auswahl von Vorträgen und Workshops:
\begin{itemize}
	\item Hacking: Interpreter für Heidenhein, einer Steuersprache für CNC-Fräsen
	\item Projektvorstellung: passwdhash
	\item Projektvorstellung: Wikileaks-Kabel-Parser
	\item Gastvortrag: C64 (Thomas Findeisen aus der M18 in Weimar
	\item Vortrag: Wanderlust, ein E-Mail-Client für Emacs
	\item Vortrag: Polyglotte Programmierung
	\item Workshop: Zabbix und Postgres
	\item Spieleabend
	\item Kurzvorstellung: moderncv, eine \LaTeX-Klasse für Lebensläufe
	\item Workshop: dn42
	\item Vortrag: Formale Begriffsanalyse
	\item Vortrag: Einführung in make
	\item Vortrag: Gentoo
	\item Vortrag: Fliegen (Fortbewegung, hauptsächlich Gleitschirm)
	\item Gastvortrag: "`Vom Aussterben bedroht: Die Universalmaschine Computer"'
		(15. November, Matthias Kirschner (FSFE)
	\item GNU-Radio Workshop
	\item Cryptoparty
\end{itemize}

\subsection*{Nerdfahrschule / Hackerfahrschule}
Veranstaltungsreihe für Studienanfänger zum Start des Wintersemesters
mit dem Ziel der Vermittlung studienrelevanter Kenntnisse im IT-Bereich
\begin{itemize}
	\item Anonym im Netz
	\item Open Source und Linux
	\item Powereditoren vi und emacs
	\item Erste Schritte auf dem Linux-Terminal
	\item \LaTeX
	\item Moderne Versionskontrollsysteme
\end{itemize}

\end{document}
